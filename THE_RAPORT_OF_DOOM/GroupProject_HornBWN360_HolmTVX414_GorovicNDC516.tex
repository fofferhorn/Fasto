\documentclass[12pt]{article}
\usepackage{amsmath} 
\usepackage[utf8]{inputenc}
\usepackage[T1]{fontenc} 
\usepackage[danish]{babel} 
\usepackage{verbatim} 
\usepackage[all]{xy} 
\usepackage{setspace}
\usepackage{graphicx}

\author{Oversættere\\\\Christoffer Horn BWN360\\Kristian Holm TVX414\\Damir Gorovic NDC516}
\title{Group Project}

\begin{document}
\maketitle
\hfil \break
\hfil \break
\hfil \break
\subsection*{Introduction}
This report covers extensions and adjustments for the Fasto compiler language in the group project. We will describe the most significant changes and adjustments we have made in the different files, and will briefly discuss some of the tests and test results.
\pagebreak
\tableofcontents
\pagebreak
\section{Getting started}
Inside the CopyConstPropFold and CodeGen we have corrected some of the minor bugs (which we're already discovered) such as for example MINUS in codegen doing addition instead of subtraction. 
\subsection{The lexer}
The lexer we were given in the start of the group project contained most of the common tokens/keywords needed. We did not need to change any data which was already there, but we needed to add some new tokens and keywords which were needed for the Fasto language (such as true, false, not and so on). The keywords we have added has been done in the following way;\\
\begin{verbatim}
 fun keyword (s, pos) =
     case s of
         "if"           => Parser.IF pos
       | "then"         => Parser.THEN pos
       | "else"         => Parser.ELSE pos
       | "bool"         => Parser.BOOL pos
       | "true"         => Parser.BOOLVAL (true, pos)
       | "false"        => Parser.BOOLVAL (false, pos)
       | "char"         => Parser.CHAR pos
       | "fun"          => Parser.FUN pos
\end{verbatim}
This is just a small example of how it is implemented, but it gives a general idea of how we have approached it. \\
Other than these keywords we also needed to add some tokens which were not known for Fasto. A small example is shown here:
\begin{verbatim}
  | `+`                 { Parser.PLUS      (getPos lexbuf) }
  | `*`                 { Parser.TIMES     (getPos lexbuf) }
  | "&&"                { Parser.AND       (getPos lexbuf) }
  | `<`                 { Parser.LTH       (getPos lexbuf) }
  | `)`                 { Parser.RPAR      (getPos lexbuf) }
  | eof                 { Parser.EOF       (getPos lexbuf) }
  | _                   { lexerError lexbuf "Illegal symbol in input" };
\end{verbatim}
The code shown above should give a decent understanding of how we have approached some of the tokens needed, and everything which we do not know.\footnote{All of the tokens which we do not know and have not defined here, we will not recognize (and are therefore not valid) and therefore raise an error} A minor note can be added about the fact that we use the " sign for AND because we need to define it as a string since it needs 2 chars to be recognized.
\subsection{The parser}
Inside of the Parser.grm file we need to define how we use the tokens which we made in the lexer. First we needed to define \textit{how} our tokens looked when they were in use, so our parser can recognize them. This was done in the following way;\\
\begin{verbatim}
%token <(int*int)> PLUS MINUS TIMES DIVIDE DEQ EQ LTH AND OR
%token <(int*int)> IF THEN ELSE LET IN INT BOOL CHAR EOF
%token <bool*(int*int)> BOOLVAL
%token <string*(int*int)> ID STRINGLIT
\end{verbatim}
The "(int*int)" figure shown above is used to get the location of where these tokens are used, so we can send a helpful error message when something goes wrong. the ID and STRINGLIT tokens are konwn as a string, which also needs to have a position in the document.\\\\
Next up we need to define the precedence of our different evaluating expressions. This was done quickly and with not much trouble, as we were told in the handed out document which expressions had which precedence and associativity.\\\\
Lastly we need to define how we use the different expressions/types/parameters and so on in our language. Here is a few examples of how this was implemented;\\
\begin{verbatim}
Exp :     Exp PLUS  Exp  { Plus ($1, $3, $2) }
        | Exp DIVIDE Exp { Divide($1, $3, $2) }
        | FILTER LPAR FunArg COMMA Exp RPAR
                        { Filter ($3, $5, (), $1) }
Type :    INT           { Int }
        | LBRACKET Type RBRACKET { Array $2 } ;                                  
\end{verbatim}
\section{Interpreting}
After the parser and lexer have been implemented and told how to handle the tokens and identifiers, we move on to the interpreter. In this document we define how to evaluate the different expressions. The following example shows some examples of how this is done;\\
\begin{verbatim}
| evalExp ( Times(e1, e2, pos), vtab, ftab ) =
        let val res1   = evalExp(e1, vtab, ftab)
            val res2   = evalExp(e2, vtab, ftab)
        in  evalBinopNum(op *, res1, res2, pos)
        end
| evalExp ( Negate(e1, pos), vtab, ftab ) =
        let val res1   = evalExp(e1, vtab, ftab)
        in  evalopNum(op ~, res1, pos)
        end
\end{verbatim}
The interesting part of these implementations are the inclusion of vtab and ftab. What this basically does is making our code dynamic, in the sense that we are able to look up our previously used functions and variables.\\
Other than these "simple" implementations, we also implemented the anonymous function (commonly known as Lambda), but we will get back to this later.
\section{Typechecking}
After we have told our
\pagebreak
\section{CodeGen}
Inside the codegen file we were given a few already working functions which translated into MIPS code without a problem, such as Plus, Minus and map. We have extended the simple math expressions within CodeGen to include Not, Negate, And, Or, Times and Divide. All of which have been tested and worked without any difficulties. All of these functions were relatively simple to implement, since Divide and Times are alike, Not and Negate are alike and so on.\\\\
The real challenge came when trying to implement the Scan and Filter functions. We knew from previous lectures and individual assignments, that Filter is similar to Map, and Scan is similar to Reduce. This helped us in the sense that we could "borrow" some of the code from the two functions Map and Reduce and adjust them.
\begin{verbatim}
val loop_beg = newName "loop_beg"
val loop_end = newName "loop_end"
val tmp_reg = newName "tmp_reg"
val loop_header = [ Mips.LABEL (loop_beg)
                  , Mips.SUB (tmp_reg, i_reg, size_reg)
                  , Mips.BGEZ (tmp_reg, loop_end) ]
\end{verbatim}
This particular piece of code (which is from our Filter function) has been used both in Map, Reduce and Scan (in \textit{slightly} different variations) since all of these functions need to have a loop. More intuitively this can be described as a "while" expression which we know from languages such as Java or C++, to tell us when to stop the loop. 
\subsection{Differences in similar functions}
Up until the point of working with the CodeGen file we did not do any compiling within our Fasto language. So far we had only used interpret mode to test that our parser/lexer/interpreter was correctly implemented. This resulted in us using the \textbf{wrong} command when compiling our code. we were almost certain that we had implemented our Filter function correctly, but we could not get it to compile, therefore we started working on the Scan function instead. Somewhere in the middle of the Scan function, we got some help from a TA, and he showed us the mistake we did not know we were making; Using the "fasto -o" command instead of "fasto -c" command. Suddenly, we could compile our Filter function into MIPS code and test it.\\\\ 
We had a single mistake of moving around information between the wrong registers which was quickly corrected. This ment though, that we could test our Scan function \textit{while} coding it, thus making it alot easier for us to discover mistakes before we went further in the code. This resulted in some slightly different syntaxes and register names within these two functions, but both work as intended and have been tested thoroughly. 

\section{CopyContPropFold}
In this part of the code we have made several intuitive and simple, yet important, optimizations using simple math expressions. It can seem very redundant to make some of these changes, but in the long run it will make our code run much faster.
\subsection{Division and Multiplication}
The most notable optimizations made in division and multiplication is the case where a 0 is included. What do we do when we have to divide or multiply a number with 0?\\
The solution is sort of trivial yet important since it saves us a lot of time. When we multiply a number with 0, it always has to return 0, simple and straight forward. With division, the only exception we have to make is when dividing with zero we have to return an error and not 0, since division with 0 is not allowed.
\begin{verbatim}
in case (e1', e2') of
       (Constant (IntVal x, _), Constant (IntVal y, _)) =>
       Constant (IntVal (x*y), pos)
     | (Constant (IntVal 0, _), _) =>
       Constant (IntVal 0, pos)
     | (_, Constant (IntVal 0, _)) =>
       Constant (IntVal 0, pos)
     | (_, Constant (IntVal 1, _)) => e1'
     | (Constant (IntVal 1, _), _) => e2'
     | _ => Times (e1', e2', pos)
\end{verbatim}
In the example shown above (which is taken from multiplication) we can see that when a 0 occurs in as one of our expressions, we simply return 0. The same is done in division, except it returns and error.
\subsection{Negate and Not}
\section{Tests}
In this following section we will show relevant tests to prove the correctness of our implementations across the Fasto language.\\
Apart from the gives tests we received with the Fasto folder (which all work and run correctly), we have implemented a few of our own to test some of the important extensions we have made.
\subsection{Precedence}
To test that our precedence of expressions are correct we have used the following program;
\begin{verbatim}
fun int main() =
write(3 + 9 - 15 * 2 / 10)
\end{verbatim}
This program returns the correct output "9". It does the multiplication first, then the division and thereafter the addition and subtraction.\\
Furthermore we have tested that our booleans are expressed correctly by making this slight adjustment to the program (which returns true);
\begin{verbatim}
fun bool main() =
write(3 + 9 - 15 * 2 / 10 == 9)
\end{verbatim}
\subsection{And/Or}
To test that our And/Or expressions are handled correctly we have used the following program;
\begin{verbatim}
fun bool main() =
write(true && true || false)
\end{verbatim}
The program returns "True" which is what we expect and want from it. If we choose to remove the and expression and do "write(true || false)" it will return false, which shows that both are expressions are working as intended.
\subsection{Not}
The program we use to test out "not" expressions looks like this;
\begin{verbatim}
fun bool main() =
write(not false)
\end{verbatim}
This returns true which is the result we expect.
\subsection{Simple typechecking}
Except for the fact that we do not get any type checking errors in the other tests, we found it necessary to make a test which would give us a type error to ensure this was implemented correctly. The test looks as follows;
\begin{verbatim}
fun bool main() =
    write(not 3)
\end{verbatim}
This program returns "Type error: Cannot unify types bool and int at line 2 column 11", which is what we want.
\subsection{Redundant testing}
This section will cover some of the more redundant test we have made to make sure that our implementations are correct. These tests will include a "double negate" and a "double not" function. Below can be seen an example of how we have made sure, that we can use double not or double negate expressions;
\begin{verbatim}
fun bool main() =
    write(not not true)
    
fun int main() =
    write(~~1)
\end{verbatim}
Both of these functions are compiled correctly and return the right result. In the first program we would like to return "true" and in the second program we want to return "1".

\pagebreak

\section{Appendix}
\subsection{A - CodeGen}
In this appendix you can see the code for different functions in the CodeGen.sml file.

\subsubsection{Filter}
Here is the code for the function scan. 

\begin{verbatim}
Filter (farg, arr_exp, elem_type, pos) =>
  let val size_reg = newName "size_reg" (* size of input/output array *)
      val arr_reg  = newName "arr_reg"  (* address of array *)
      val elem_reg = newName "elem_reg" (* address of single element *)
      val res_reg  = newName "res_reg"  (* result register *)
      val arr_code = compileExp arr_exp vtable arr_reg

      val get_size = [ Mips.LW (size_reg, arr_reg, "0") ] 
      (* Gets the size of the input array *)
      
      val alloc_space = dynalloc(size_reg, place, elem_type)

      val i_reg    = newName "i_reg"    (* the invariant i *)
      val j_reg = newName "j_reg"
      val addr_reg = newName "addr_reg" 
      (* address of element in new array *)

      val init_regs = [ Mips.ADDI (addr_reg, place, "4")
                      , Mips.MOVE (i_reg, "0")
                      , Mips.ADDI (elem_reg, arr_reg, "4") ]

      val loop_beg = newName "loop_beg"
      val loop_end = newName "loop_end"
      val tmp_reg = newName "tmp_reg"
      val loop_header = [ Mips.LABEL (loop_beg)
                        , Mips.SUB (tmp_reg, i_reg, size_reg)
                        , Mips.BGEZ (tmp_reg, loop_end) ]

(* filter is 'if f(old_arr[i]) = true then arr[j] = old_arr[i]'. *)
      val arr_val = newName "arr_val"
      val loop_filter0 =
          case getElemSize elem_type of
              One  => Mips.LB(arr_val, elem_reg, "0")
                      :: applyFunArg(farg, [arr_val], 
                      vtable, res_reg, pos)
                      @ [ Mips.ADDI(elem_reg, elem_reg, "1") ]
                      @ [ Mips.ADDI (i_reg, i_reg, "1") ]
            | Four => Mips.LW(arr_val, elem_reg, "0")
                      :: applyFunArg(farg, [arr_val], 
                      vtable, res_reg, pos)
                      @ [ Mips.ADDI(elem_reg, elem_reg, "4") ]
                      @ [ Mips.ADDI (i_reg, i_reg, "1") ]

      val branch = [ Mips.BEQ(res_reg, "0", loop_beg) ]

      val loop_filter1 =
          case getElemSize elem_type of
              One  => [ Mips.SB (arr_val, addr_reg, "0") ]
            | Four => [ Mips.SW (arr_val, addr_reg, "0") ]

      val loop_footer =
     [ Mips.ADDI (addr_reg, addr_reg,
            makeConst (elemSizeToInt (getElemSize elem_type)))
     , Mips.ADDI (j_reg, j_reg, "1")
     , Mips.J loop_beg
     , Mips.LABEL loop_end
     , Mips.SW (j_reg, place, "0")
     ]

  in
    arr_code
    @ get_size
    @ alloc_space
    @ init_regs
    @ loop_header
    @ loop_filter0
    @ branch
    @ loop_filter1
    @ loop_footer
  end
\end{verbatim}
\hfill
\hfill
\subsubsection{Scan}
\begin{verbatim}
Scan (binop, acc_exp, arr_exp, tp, pos) =>
let val arr_reg  = newName "arr_reg"   (* address of input array *)
    val size_reg = newName "size_reg"  (* size of input/output array *)
    val i_reg    = newName "i_reg"     (* loop counter *)
    val tmp_reg  = newName "tmp_reg"   (* several purposes *)
    val res_reg  = newName "res_reg"   
    (* result register of function calls *)
    val addr_reg = newName "addr_reg"

    val loop_beg = newName "loop_beg"
    val loop_end = newName "loop_end"

    val arr_code = compileExp arr_exp vtable arr_reg

    val get_size = [ Mips.LW (size_reg, arr_reg, "0") ] 
    (* Gets the size of the input array *)

    val alloc_space = [ Mips.ADDI (size_reg, size_reg, "1") ]
                      @ dynalloc(size_reg, place, tp)

    (* Compile initial value into 0[place] (will be updated below) *)
    val acc_code = compileExp acc_exp vtable res_reg 

    val save_init = case getElemSize tp of
              One  => [ Mips.SB (res_reg, place, "1")
                      , Mips.ADDI (addr_reg, place, "1")
                      , Mips.ADDI (arr_reg, arr_reg, "1")]
            | Four => [ Mips.SW (res_reg, place, "4")
                      , Mips.ADDI (addr_reg, place, "4")
                      , Mips.ADDI (arr_reg, arr_reg, "4")]

    (* Set i_reg to 0. While i < size_reg, loop. *)
    val loop_code =
        [ Mips.MOVE (i_reg, "0")
        , Mips.LABEL (loop_beg)
        , Mips.SUB (tmp_reg, i_reg, size_reg)
        , Mips.BGEZ (tmp_reg, loop_end) ]

      (* Load arr[i] into tmp_reg *)
    val load_code =
        case getElemSize tp of
            One =>  Mips.LB   (tmp_reg, arr_reg, "0")
                    :: applyFunArg(binop, [res_reg, tmp_reg], 
                    vtable, res_reg, pos)
                    @ [ Mips.ADDI (arr_reg, arr_reg, "1") ]
                    @ [ Mips.ADDI (addr_reg, addr_reg, "1") ]
          | Four => Mips.LW   (tmp_reg, arr_reg, "0")
                    :: applyFunArg(binop, [res_reg, tmp_reg], 
                    vtable, res_reg, pos)
                    @ [ Mips.ADDI (arr_reg, arr_reg, "4") ]
                    @ [ Mips.ADDI (addr_reg, addr_reg, "4") ]

    val save_to_arr =
        case getElemSize tp of
                      One  => [ Mips.SB(res_reg, addr_reg, "0") ]
                    | Four => [ Mips.SW(res_reg, addr_reg, "0") ]

    val loop_footer =
        [ Mips.ADDI (i_reg, i_reg, "1")
        , Mips.J loop_beg
        , Mips.LABEL loop_end
        ]

  in   arr_code 
     @ get_size
     @ alloc_space
     @ acc_code 
     @ save_init
     @ loop_code 
     @ load_code 
     @ save_to_arr
     @ loop_footer
  end
\end{verbatim}

\end{document}